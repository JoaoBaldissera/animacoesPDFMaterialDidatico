\documentclass{article}
\usepackage[a4paper,left=2cm,right=2cm,top=2cm,bottom=3cm]{geometry}
\usepackage[utf8]{inputenc}
\usepackage[portuguese]{babel}
\usepackage{graphicx}
\usepackage{animate}
\usepackage{float}
\usepackage{amsmath}
\usepackage{steinmetz}
\usepackage{hyperref}
\usepackage[framed]{matlab-prettifier}

\title{Exemplos artigo}
\author{João Victor Omar Baldissera \\ Roberto Kawakami Harrop Galvão \\ José Roberto Colombo Júnior \\ Instituto Tecnológico de Aeronáutica, Electronics Engineering Division \\ 12228-900 São José dos Campos, SP, Brasil}

\date{28 de março de 2022}

\begin{document}

% \maketitle

\section*{Exemplo}

\noindent
Considere o seguinte modelo para um sistema massa-mola-amortecedor:
\begin{equation*}
    \underbrace{\left[ \begin{matrix} \dot{x}_{1} \\ \dot{x}_{2} \end{matrix} \right]}_{\dot{x}(t)} = \underbrace{\left[ \begin{matrix} 0 & 1 \\ -9 & -2 \end{matrix} \right]}_{A} \underbrace{\left[ \begin{matrix} x_{1} \\ x_{2} \end{matrix} \right]}_{x(t)} + \underbrace{\left[ \begin{matrix} 0 \\ 1 \end{matrix} \right]}_{B} u(t),
\end{equation*}
\begin{equation*}
    y = \underbrace{\left[ \begin{matrix}1 & 0 \end{matrix} \right]}_{C} \left[ \begin{matrix} x_{1} \\ x_{2} \end{matrix} \right],
\end{equation*}
em que $x_{1}$ denota a deformação da mola em metro e $x_{2}$ a taxa de deformação da mola em metro por segundo. Considerando que seja possível medir todos os estados do sistema, projetou-se um sistema de regulação empregando o LQR (Regulador Linear Quadrático) com os parâmetros
\begin{equation*}
    Q = \left[ \begin{matrix} 100 & 0 \\ 0 & 1 \end{matrix} \right] \text{ e } R = 1.
\end{equation*}
Para isso, considerou-se a lei de controle
\begin{equation*}
    u(t) = -Kx(t)
\end{equation*}
e o problema de otimização 
\begin{equation*}
\begin{array}{rl}
    \text{minimizar} & J = \int_{0}^{\infty} \left[x^{T}(t)Qx(t) + u^{T}(t)Ru(t)\right]~dt \\
    \text{sujeito a} & \dot{x}(t) = Ax(t) + Bu(t)
\end{array}
\end{equation*}
cuja solução é obtida resolvendo-se a Equação Algébrica de Riccati
\begin{equation*}
    A^{T}P + PA - PBR^{-1}B^{T}P + Q = 0
\end{equation*}
e fazendo $K = R^{-1}B^{T}P$. Utilizando o comando \textit{lqr} do MATLAB\textsuperscript{\textregistered{}}, obteve-se $K = \left[ \begin{matrix} 4,4536 & 1,7292 \end{matrix} \right]$.

\begin{center}
    \textbf{Simulação}
\end{center}

\noindent
A simulação foi realizada no MATLAB\textsuperscript{\textregistered{}} (Apêndice) através da discretização exata do sistema (\textit{lsim}) com passo $\delta{}t = 0,03~\text{s}$.

\begin{figure}[H]
    \centering
    \animategraphics[controls,width=\textwidth{}]{203}{Exemplo/simulacao}{0}{202}
    \caption{Representação animada do controle de um sistema massa mola amortecedor utilizando o LQR}
    \label{fig:massaMola}
\end{figure}

Na Figura \ref{fig:massaMola}, existem 203 imagens sobrepostas, nomeadas sequencialmente como (\textit{simulacao0.pdf}), (\textit{simulacao1.png}), $\dots$, (\textit{simulacao201.png}) e (\textit{simulacao202.pdf}). Vale ressaltar que para reduzir o tamanho do arquivo PDF, a primeira e última figura encontram-se no formato \textbf{.pdf} e as demais no formato \textbf{.png}. Além disso, é possível interagir com a animação através dos comandos

\begin{table}[H]
    \centering
    \begin{tabular}{rl}
    \includegraphics[scale=0.2]{Comandos/recuoParaInicio.png} & Recuo total para o início  \\
    \includegraphics[scale=0.2]{Comandos/recuoManual.png} & Recuo manual \\
    \includegraphics[scale=0.2]{Comandos/recuoAutomatico.png} & Recuo automático \\
    \includegraphics[scale=0.2]{Comandos/avancoAutomatico.png} & Avanço automático \\
    \includegraphics[scale=0.2]{Comandos/avancoManual.png} & Avanço manual \\
    \includegraphics[scale=0.2]{Comandos/avancoParaFinal.png} & Avanço total para o final \\
    \includegraphics[scale=0.2]{Comandos/aumentaVelocidadeAutomatico.png} & Aumenta a velocidade de transição do modo automático \\
    \includegraphics[scale=0.2]{Comandos/padraoVelocidadeAutomatico.png} & Velocidade padrão de transição do modo automático \\
    \includegraphics[scale=0.2]{Comandos/diminuiVelocidadeAutomatico.png} & Diminui a velocidade de transição do modo automático
    \end{tabular}
\end{table}

\noindent
Finalmente, um exemplo de código para ser utilizado na inserção destas animações no \LaTeX:

\newpage

\begin{verbatim}

\documentclass{article}
\usepackage[a4paper,left=2cm,right=2cm,top=2cm,bottom=3cm]{geometry}
\usepackage[utf8]{inputenc}
\usepackage[portuguese]{babel}
\usepackage{graphicx}
\usepackage{animate}
\usepackage{float}

\newcommand{\iTotal}{203}                   % Há 203 figuras,
\newcommand{\sEndereco}{Exemplo1/simulacao} % localizadas na pasta Exemplo1.
\newcommand{\iInicial}{0}                   % A primeira, nomeada como simulacao0,
\newcommand{\iFinal}{202}                   % e a última como simulacao202

\begin{document}

\begin{figure}[H]
    \centering
    \animategraphics[controls,width=\textwidth{}]{\iTotal}{\sEndereco}{\iInicial}{\iFinal}
    \caption{Representação animada}
\end{figure}

\end{document}
\end{verbatim}

\ 

\section*{Apêndice}

\noindent
Destaca-se o código MATLAB\textsuperscript{\textregistered{}} que foi utilizado para o desenvolvimento deste exemplo.

\begin{lstlisting}[style=Matlab-editor]
A = [0 1; -9 -2]; n = length(A);
B = [0; 1]; p = size(B,2);
C = [1 0;0 1]; q = size(C,1);
D = 0;
Q = diag([100,1]);
R = 1;
K = lqr(A,B,Q,R);

Planta = ss(A,B,C,0,'InputName','u','OutputName',{'x1','x2'});
Controlador = tf(-K,'InputName',{'x1','x2'},'OutputName','u');
Gmf = connect(Planta,Controlador,'u',{'x1','x2'});

tFin=6;
deltaT=0.03;
t = 0:deltaT:tFin;
x0 = [-10;0];

% Simulacao em malha aberta
uMA = zeros(1,length(t));
xMA = lsim(Planta,uMA,t,x0); xMA = xMA';

% Simulacao em malha fechada
uMF = zeros(1,length(t));
xMF = lsim(Gmf,uMF,t,x0); xMF = xMF';
for i=1:length(t)
    uMF(i) = -K*xMF(:,i);
end

for it = 1:length(t)
    plotResultados(1,t,xMF,xMA,uMF,uMA,it)
    set(gcf, 'Color', 'w')
    export_fig(['simulacao',num2str(it)],'-png','-q101','-nocrop')
end

export_fig('simulacao0','-pdf','-q101','-nocrop')
export_fig(['simulacao',num2str(it+1)],'-pdf','-q101','-nocrop')
\end{lstlisting}

\ 

\noindent
E também a função que foi utilizada para gerar cada uma das 203 figuras.

\begin{lstlisting}[style=Matlab-editor]
function plotResultados(figura,t,xMF,xMA,uMF,uMA,it)

FIGURA = figure(figura);
FIGURA.Units = 'centimeters'; FIGURA.Position = [0 0 20 16];

subplot(2,2,1)
    MASSAMF = [3 3 -3 -3 3; 0 1 1 0 0]; 
	MASSAMA = [3 3 -3 -3 3; 0 1 1 0 0];
	MOLAMF(1,:) = linspace(-15,xMF(1,it)-3,10);
    MOLAMF(2,:) = 0.7+0.1.*[1 -1 1 -1 1 -1 1 -1 1 -1];
	MOLAMA(1,:) = linspace(-15,xMA(1,it)-3,10);
    MOLAMA(2,:) = 0.7+0.1.*[1 -1 1 -1 1 -1 1 -1 1 -1];
    aux = linspace(-15,xMF(1,it)-3,4);
    AMORTMF(1,:) = [aux(1) aux(2) aux(2) aux(3) aux(3) aux(2) aux(2) ...
        aux(3) aux(3) aux(4)];
    AMORTMF(2,:) = 0.3+0.1.*[0 0 1 1 -1 -1 1 1 0 0];
	aux = linspace(-15,xMA(1,it)-3,4);
	AMORTMA(1,:) = [aux(1) aux(2) aux(2) aux(3) aux(3) aux(2) aux(2) ...
        aux(3) aux(3) aux(4)];
    AMORTMA(2,:) = 0.3+0.1.*[0 0 1 1 -1 -1 1 1 0 0];
	plot([0 0],[-0.5 1.5],'color',0.7.*[1 1 1],'LineWidth',2); hold on;
    plot([-15 7],[0 0],'color',0.7.*[1 1 1],'LineWidth',2);
	plot(MASSAMF(1,:)+xMF(1,it),MASSAMF(2,:),'blue','LineWidth',2);
    plot(MOLAMF(1,:),MOLAMF(2,:),'blue','LineWidth',2);
	plot(AMORTMF(1,:),AMORTMF(2,:),'blue','LineWidth',2);
    plot(MASSAMA(1,:)+xMA(1,it),MASSAMA(2,:),'--red','LineWidth',1); 
    plot(MOLAMA(1,:),MOLAMA(2,:),'--red','LineWidth',1);
    plot(AMORTMA(1,:),AMORTMA(2,:),'--red','LineWidth',1); hold off;
    AX=gca; AX.XLim = [-15 7]; AX.YLim = [-0.5 1.5]; AX.YTickLabel = [];
    AX.XTick = [-10 0];
    xlabel('Posi\c{c}\~{a}o (metro)','Interpreter','latex');
    AX = gca;
    text(max(AX.XLim), max(AX.YLim),'(a)',...
        'HorizontalAlignment','right', ...
        'VerticalAlignment','top', ...
        'Interpreter','latex');
    text(min(AX.XLim), max(AX.YLim),...
        ['Tempo: ',strrep(num2str(round(t(it),2),'%.2f'),'.',','),' s'],...
        'HorizontalAlignment','left', ...
        'VerticalAlignment','top', ...
        'Interpreter','latex');

subplot(2,2,2)
    plot([0 t(end)],[0 0],'color',0.7.*[1 1 1],'LineWidth',2); hold on;
    MF = plot(t(1:it),xMF(1,1:it),'blue','LineWidth',2);
    MA = plot(t(1:it),xMA(1,1:it),'--red','LineWidth',2); hold off;
    AX=gca; AX.YLim = [min(min(xMF(1,:)),min(xMA(1,:)))-3,...
        max(max(xMF(1,:)),max(xMA(1,:)))+3];
    xlabel('Tempo (segundo)','Interpreter','Latex'); 
    ylabel('Posi\c{c}\~{a}o (metro)','Interpreter','Latex');
    LEGENDA = legend([MF,MA],'Malha fechada','Malha aberta');
    LEGENDA.Interpreter = 'Latex';
    LEGENDA.Position = [0.4335    0.4779    0.1722    0.0604];
    AX = gca;
    text(max(AX.XLim), max(AX.YLim),'(b)',...
        'HorizontalAlignment','right', ...
        'VerticalAlignment','top', ...
        'Interpreter','latex');

subplot(2,2,3)
    plot([0 t(end)],[0 0],'color',0.7.*[1 1 1],'LineWidth',2); hold on;
    MF = plot(t(1:it),uMF(1:it),'blue','LineWidth',2);
    MA = plot(t(1:it),uMA(1:it),'--red','LineWidth',2); hold off; 
    AX=gca; AX.YLim = [min(min(uMF),min(uMA))-3, ...
        max(max(uMF),max(uMA))+3];
    xlabel('Tempo (segundo)','Interpreter','Latex'); 
    ylabel('Controle (newton)','Interpreter','Latex');
    AX = gca;
    text(max(AX.XLim), max(AX.YLim),'(c)',...
        'HorizontalAlignment','right', ...
        'VerticalAlignment','top', ...
        'Interpreter','latex');
    
subplot(2,2,4)
    plot([0 t(end)],[0 0],'color',0.7.*[1 1 1],'LineWidth',2); hold on;
    MF = plot(t(1:it),xMF(2,1:it),'blue','LineWidth',2);
    MA = plot(t(1:it),xMA(2,1:it),'--red','LineWidth',2); hold off;
    AX=gca; AX.YLim = [min(min(xMF(2,:)),min(xMA(2,:)))-3, ...
        max(max(xMF(2,:)),max(xMA(2,:)))+3];
    xlabel('Tempo (segundo)','Interpreter','Latex'); 
    ylabel('Velocidade (metro/segundo)','Interpreter','Latex');
    AX = gca;
    text(max(AX.XLim), max(AX.YLim),'(d)',...
        'HorizontalAlignment','right', ...
        'VerticalAlignment','top', ...
        'Interpreter','latex');
end
\end{lstlisting}

\ 

\noindent
Vale ressaltar que tambem é necessário instalar o programa \textbf{export\_fig}, disponível em
\begin{center}
    \url{https://www.mathworks.com/matlabcentral/fileexchange/23629-export_fig},
\end{center}
o programa \textbf{Postscript and PDF interpreter/renderer}, disponível em
\begin{center}
    \url{https://ghostscript.com/releases/index.html}
\end{center}
e o programa \textbf{Xpdf command line tools}, disponível em
\begin{center}
    \url{https://www.xpdfreader.com/download.html}
\end{center}

\end{document}